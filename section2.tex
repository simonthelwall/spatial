\documentclass[11pt,a4paper]{article}
\usepackage[latin1]{inputenc}
\usepackage[british]{babel}
\usepackage{amsmath}
\usepackage{amsfonts}
\usepackage{amssymb}
\usepackage{graphicx}
\author{106936}
\title{A critical synthesis of modern methods in spatial statistics - point processes}
\begin{document}
\maketitle
%\begin{enumerate}
%\item Describe a spatial point process (20 \%)
%\item Using examples of at least two disease models, discuss advantages and disadvantages of cluster detection methods and application to inf dis control programmes. (80 \%)
%\end{enumerate}
\section{Spatial point processes}
Spatial data can be classified into three forms: point, continuous and aggregated. 
Aggregated data combines multiple measures of some exposure or outcome into an overall value for some defined geographical area. 
These data can then be analysed based on their spatial neighbourhood. 
As with non-spatial data, continuous data can take any value within a range. 
In contrast to these forms, point data is the occurrence of a discrete event within space. 

A spatial point process is a mathematical description of the spatial distribution of events occurring on a plane. 
The number of points (events) and the positions of points on the plane are are subject to variation due to chance, i.e. they are random. 
The analysis of spatial point processes aims to estimate the parameters of the process. 

For example, the locations of diagnoses of people with lymphatic filariasis will be distributed at random throughout space. 
Each diagnosis would be an event, and the location of each event would be subject to random variation. 
However, the frequency with which events occur will vary according a number of factors, including the density of parasite vectors in the local environment. 
These are first order effect effects and are deterministic. 
In addition to the deterministic effects, second order effects may be present. 
In second order effects there is spatial dependency and events are either more likely to occur in areas where there are already a high number of the same event, or less likely to occur in such areas. 
The former case is more likely to occur in communicable disease epidemiology, where a person is more likely to become infected when proximate to other people who are already infected. 

\section{Advantages and disadvantages of cluster detection methods and their application to infectious disease control programs}

For the purposes of infectious disease epidemiology, clusters are areas where disease levels (either incidence or prevalence) are higher or lower than might be expected by chance.
Cluster detection methods make use of spatial point processes, and the implicit null hypothesis that points are evenly distributed in space, to identify areas of unusually high or low disease levels.
The use of such techniques has two potential purposes: to test hypotheses that certain factors, unevenly distributed in space, contribute to the risk of of a particular disease; and to identify regions with high levels of disease in order to assign greater funding or effort in reducing such levels.

\subsection{Methods for cluster detection}
A large number of statistical methods exist for the detection of spatial clustering. 
A recent systematic review found that Ripley's K-function was the most commonly used method for the detection of global clustering. \ref{Fritz2013}
Ripley's K-function compares the distribution of a pattern of points to that generated by a homogenous Poisson point process. 

The K-function has been used to \ldots
Kulldorff's spatial scan statistic uses an expanding window, varying in position, to compare levels of disease inside or outside the window to identify areas that are considerably higher or lower than expected. 
In a comparison, the spatial scan statistic was as sensitive as kernel methods in predicting future prevalence of malaria, and more sensitive than weighted local prevalence scoring methods. \ref{Mosha2014}

\subsection{Limitations of cluster detection}
In some ways, the limitations of cluster detection are the same as those of standard epidemiological techniques. 
Cluster detection is at least equally prone to errors of chance, bias and confounding. 
Systematic errors in the reporting or recording of spatial data in disease occurence will bias data, resulting in the detection of more or fewer clusters than there really are. 

Measurement error of spatial data can occur in two ways. 
Either an instrument can be prone to greater variation in the recording of location or it can be consistantly inaccurate in the measurement of location and offset the measurement of the true location of a point by the same amount each time. 
The first form of measurement error will weaken both global and local measures of clustering as points will be more evenly distributed than they truely are. 
Unpicking the consequences of the second form of measurement error is more complex. 
If global or local clustering is being considered alone, then there will be no consequence and points will remain the same distance from each other, resulting in the same measures of clustering that would otherwise be detected.
However, if the locations of clusters matter, either on their own or in relation to the presence of other factors, such as breeding grounds for filarial vectors, then any control programmes will be based on erroneous information and potentially not administered where necessary. 

As with other epidemiological techniques the assumptions behind statistical methods need to be examined before the techniques are applied. 
Where methods apply Poisson point processes, there is the assumption inherent in the Poisson model that the occurrence of one event does not affect the probability of occurrence of another. 
This may be true for the non-communicable disease (though not necessarily even then) but is less likely to be true for communicable disease, where an individual is more likely to contract a disease when in an area of high prevalence. \ref{Kirkwood2003}

Similarly, cluster detection methods are vulnerable to manipulation depending on the parameters that are set when running tests, 
\end{document}