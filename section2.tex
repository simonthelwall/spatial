\documentclass[11pt,a4paper]{article}
\usepackage[latin1]{inputenc}
\usepackage[british]{babel}
\usepackage{amsmath}
\usepackage{amsfonts}
\usepackage{amssymb}
\usepackage{graphicx}
\usepackage{natbib}
\author{106936}
\title{A critical synthesis of modern methods in spatial statistics - point processes}
\begin{document}
\maketitle
%\begin{enumerate}
%\item Describe a spatial point process (20 \%)
%\item Using examples of at least two disease models, discuss advantages and disadvantages of cluster detection methods and application to inf dis control programmes. (80 \%)
%\end{enumerate}
\section{Spatial point processes}
Spatial data can be classified into three forms: point, continuous and aggregated. 
Aggregated data combines multiple measures of some exposure or outcome into an overall value for some defined geographical area. 
These data can then be analysed based on their spatial neighbourhood. 
As with non-spatial data, continuous data can take any value within a range. 
In contrast to these forms, point data is the occurrence of a discrete event within space. 

A spatial point process is a mathematical description of the spatial distribution of events occurring on a plane. 
The number of points (events) and the positions of points on the plane are are subject to variation due to chance, i.e. they are random. 
The analysis of spatial point processes aims to estimate the parameters of the process. 

For example, the locations of diagnoses of people with lymphatic filariasis will be distributed at random throughout space. 
Each diagnosis would be an event, and the location of each event would be subject to random variation. 
However, the frequency with which events occur will vary according a number of factors, including the density of parasite vectors in the local environment. 
These are first order effect effects and are deterministic. 
In addition to the deterministic effects, second order effects may be present. 
In second order effects there is spatial dependency and events are either more likely to occur in areas where there are already a high number of the same event, or less likely to occur in such areas. 
The former case is more likely to occur in communicable disease epidemiology, where a person is more likely to become infected when proximate to other people who are already infected. 

\section{Advantages and disadvantages of cluster detection methods and their application to infectious disease control programs}

For the purposes of infectious disease epidemiology, clusters are areas where disease levels (either incidence or prevalence) are higher or lower than might be expected by chance.
Cluster detection methods make use of spatial point processes, and the implicit null hypothesis that points are evenly distributed in space, to identify areas of unusually high or low disease levels.
The use of such techniques has two potential purposes: to test hypotheses that certain factors, unevenly distributed in space, contribute to the risk of of a particular disease; and to identify regions with high levels of disease in order to assign greater funding or effort in reducing such levels.

\subsection{Methods for cluster detection}
A large number of statistical methods exist for the detection of spatial clustering. 
A recent systematic review found that Ripley's K-function was the most commonly used method for the detection of global clustering. \cite{Fritz2013}
Ripley's K-function assumes a completely mapped spatial point process (i.e. all events are known and have spatial data) compares the distribution of a pattern of points to that generated by a homogeneous Poisson point process. \cite{Dixon2002} 

Eberhart \textit{et al} used the K-function to identify clusters of failure to engage in HIV care in Philadelphia and further used multivariable logistic regression to identify factors that were associated with such failures to engage. \cite{Eberhart2013}

Kulldorff's spatial scan statistic uses an expanding window, varying in position, to compare levels of disease inside or outside the window to identify areas that are considerably higher or lower than expected.\cite{Kulldorff1995} 
In a comparison, the spatial scan statistic was as sensitive as kernel methods in predicting future prevalence of malaria, and more sensitive than weighted local prevalence scoring methods. \cite{Mosha2014}

\subsection{Limitations of cluster detection}
In some ways, the limitations of cluster detection are the same as those of standard epidemiological techniques. 
Cluster detection is at least equally prone to errors of chance, bias and confounding. 
Systematic errors in the reporting or recording of spatial data in disease occurrence will bias data, resulting in the detection of more or fewer clusters than there really are. 

Measurement error of spatial data can occur in two ways. 
Either an instrument can be prone to greater variation in the recording of location or it can be consistently inaccurate in the measurement of location and offset the measurement of the true location of a point by the same amount each time. 
The first form of measurement error will weaken both global and local measures of clustering as points will be more evenly distributed than they truly are. 
Unpicking the consequences of the second form of measurement error is more complex. 
If global or local clustering is being considered alone, then there will be no consequence and points will remain the same distance from each other, resulting in the same measures of clustering that would otherwise be detected.
However, if the locations of clusters matter, either on their own or in relation to the presence of other factors, such as breeding grounds for filarial vectors, then any control programmes will be based on erroneous information and potentially not administered where necessary. 

As with other epidemiological techniques the assumptions behind statistical methods need to be examined before the techniques are applied. 
Where methods apply Poisson point processes, there is the assumption inherent in the Poisson model that the occurrence of one event does not affect the probability of occurrence of another. 
This may be true for the non-communicable disease (though not necessarily even then) but is less likely to be true for communicable disease, where an individual is more likely to contract a disease when in an area of high prevalence. \cite{Kirkwood2003}

In common with other statistical techniques, spatial cluster analysis is vulnerable to  multiple comparisons resulting in spurious associations being detected. \cite{Olsen1996}
Multiple comparisons might arise in two potential situations. 
Having detected a cluster of disease, the temptation would then be to hypothesise an environmental cause for the location of the cluster. 
Given the potential richness in environmental risks factors, comparisons could continue until a statistically significant association is found. 
In the second scenario, rates of disease of a study area could be aggregated by some administrative area, and then clustering performed. 
If there were 20 or more administrative areas, the clustering analysis would have performed twenty comparisons and false positives might have been detected. 

%%%%%%%%%%%%%%%%%%%%%%%%%%%%%%%%%%%%%%%%%
% Need to finish this para, not great starting with similarly either
%%%%%%%%%%%%%%%%%%%%%%%%%%%%%%%%%%%%%%%%%
Similarly, cluster detection methods are vulnerable to manipulation depending on the parameters that are set when running tests, adjusting the population to be covered in the spatial scan statistic might 
%%%%%%%%%%%%%%%%%%%%%%%%%%%%%%%%%%%%%%%%%

In the investigation of non-communicable diseases there is a risk that in the response to a hypothesised environmental risk factor for disease, the analyst runs the risk of the texas sharpshooter problem. \cite{Rothman1990}\cite{Elliott2004}
Such investigations would run the risk of a flexible boundary for the study area, prior beliefs and recall biasing case finding and exposure status. 
This could equally apply to infectious disease where disease might be spread by contaminated water, such as cholera or leptospirosis. 

A particular limitation of a number of cluster detection methods, including the spatial scan statistic, is the common requirement for knowledge of the spatial distribution of the population in which clusters occur. \cite{Kulldorff1995}
Accurate or complete population data may not always be available when analysing spatial data.

As described above the spatial scan statistic uses circles to identify clusters. 
Clusters are not likely to be circular, and the reliance of circles in comparing relative levels of disease may be a limitation. 
New methods have circumvented this limitation allowing flexible shapes to be used in cluster detection. \cite{Tango2005}


\subsection{Advantages of spatial cluster detection}
In a situation of high and evenly distributed prevalence, cluster detection is of limited utility.
All geographic areas require action and the identification of (even) higher prevalence areas adds little.
In contrast, when a geographical region has generally low prevalence of a disease then cluster detection can facilitate the appropriate targeting of resources to eliminate the remaining risk factors. 

Cluster analysis has been proposed as a method of prospective surveillance for infectious diseases.
An analysis of clustering of bird deaths found that detection of clusters preceded human cases of West Nile Virus with a median of 12 days before the onset of illness and 17 before diagnosis. \cite{Mostashari2003}
Although such a surveillance programme would not necessarily help prevent human cases, it would serve to alert local health services of the likelihood of cases in their area, increasing awareness and promoting more rapid diagnosis and treatment. 

\bibliographystyle{plainnat}
\bibliography{sp_refs}
\end{document}