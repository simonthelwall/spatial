\documentclass[11pt,a4paper]{article}
\usepackage[latin1]{inputenc}
\usepackage[british]{babel}
\usepackage{amsmath}
\usepackage{amsfonts}
\usepackage{amssymb}
\usepackage{graphicx}
\author{106936}
\title{A critical synthesis of modern methods in spatial statistics - point processes}
\begin{document}
\maketitle
%\begin{enumerate}
%\item Describe a spatial point process (20 \%)
%\item Using examples of at least two disease models, discuss advantages and disadvantages of cluster detection methods and application to inf dis control programmes. (80 \%)
%\end{enumerate}
\section{Describe a spatial point process}
Spatial data can be classified into three forms: point, continuous and aggregated. 
Aggregated data combines multiple measures of some exposure or outcome into an overall value for some defined geographical area. 
These data can then be analysed based on their spatial neighbourhood. 
As with non-spatial data, continuous data can take any value within a range. 
In contrast to these forms, point data is the occurrence of a discrete event within space. 

A spatial point process is a mathematical description of the spatial distribution of events occurring on a plane. 
The number of points (events) and the positions of points on the plane are are subject to variation due to chance, i.e. they are random. 
The analysis of spatial point processes aims to estimate the parameters of the process. 

For example, the locations of diagnoses of people with lymphatic filariasis will be distributed at random throughout space. 
Each diagnosis would be an event, and the location of each event would be subject to random variation. 
However, the frequency with which events occur will vary according a number of factors, including the density of parasite vectors in the local environment. 
These are first order effect effects and are deterministic. 
In addition to the deterministic effects, second order effects may be present. 
In second order effects there is spatial dependency and events are either more likely to occur in areas where there are already a high number of the same event, or less likely to occur in such areas. 
The former case is more likely to occur in communicable disease epidemiology, where a person is more likely to become infected when proximate to other people who are already infected. 

\section{Discuss advantages and disadvantages of cluster detection methods and application to inf dis control programmes.}

Cluster detection methods make use of spatial point processes and 

%%%% The following may be relevant, but isn't very good. 
%Spatial point processes may often be modelled as Poisson distributions in which the distribution of counts have a mean $\lambda$ and a variance equal to the mean.  
%However, the Poisson distribution assumes that events occur independently of each other, which may not be a valid assumption for communicable diseases. \ref{Kirkwood2003}
% Example of spatial point process here. 

\end{document}